
\documentclass[12pt, draft]{article}

\usepackage{xeCJK}
\setCJKmainfont{AR PL UKai TW}

\usepackage{csquotes}
\usepackage[english]{babel}

\usepackage[notes, backend=biber]{biblatex-chicago}
\bibliography{main}

%\usepackage{parskip}

\usepackage{color}
\usepackage[usenames,dvipsnames]{xcolor}

\usepackage[normalem]{ulem}

\setmonofont{APL385}

\begin{document}

\title{TEIming the Dragon}
\author{David Höppner and Manuel Sassmann}
\date{\today}
\maketitle

\begin{abstract}
	We present a simple ontology based automatic tagging strategy
	for
	literary chinese. We exploid the
	limited linguistic variance and uniform cultural usage
	in the canonical sources to arrive at
	automatic TEI annotations of many language constructions.
	This preprocessing eases later finer-grained manual tagging.
\end{abstract}

Literary chinese \emph{wényánwén} 文言文, 
the (artificial written) language of the educated gentry-elite
of pre-modern China, was in practical use since the middle of the Warring States
period Zhànguó 戰國 (476--221) till the
abandonment of the imperial examination system in 1905.
If one would look for a modern but quite more artificial equivalent,
%In its conciseness and simple construction
literary chinese could be easily compared
to the APL (A Programming Language) notation Kenneth E. Iverson (1920--2004) invented
in 1957 to improve upon the deficiencies of the standard mathematical notion.\footnote{
	\cite{Iverson:1962}. For example the one-liner \texttt{(∼R∈R∘.×R)/R←1↓⍳R } calculates all prime numbers less then \texttt{R} by XXX.
}
Concise and of simple construction,
 both languages share a similar intrinsic beauty.
 But togther they also suffer from the same weakness that the ``to the point''
 character entails. The
 incomprehensibility of the artefacts writing it them with the passing of time.
In the case of literary chinese we have less problems with
the diciphering of grammatical constructions of a sentence,
than with the missing background knowledge and specific context
of a given text.


In the largest editorial project of late imperial china
the Qianlong emperor 乾隆 (r. 1735--1796) ordered in XXX
to bring together and edit all existent 

\begin{quote}
	Summer rain is holding on for weeks,
	[a sign that] early autumn did indeed arrive.
	Since the second lunar month last year
	when I took a leave [from office] because of illness
	till now it has been 600 days.
	Yuanming [Tao Qian] wrote:
	»«\footcite[XXX]{Hightower:1970}

	暑雨連旬
	初秋就道
	自去年二月引疾乞休
	及是六百日矣
	淵明云行行循歸路
	計日望舊居
	而今而後
	歲月庶為我有乎\footcite[1221]{JYTSJ}
\end{quote}

\end{document}
