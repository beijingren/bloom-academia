
\documentclass[12pt, draft]{article}

\usepackage{xeCJK}
\setCJKmainfont{AR PL UKai TW}

\usepackage{csquotes}
\usepackage[english]{babel}

\usepackage[notes, backend=biber]{biblatex-chicago}
\bibliography{main}

%\usepackage{parskip}

\usepackage{color}
\usepackage[usenames,dvipsnames]{xcolor}

\usepackage[normalem]{ulem}
\usepackage{listings}
\usepackage{hyperref}

\begin{document}

\title{Karl-Steinbuch-Stipendium}
\author{David Höppner and Manuel Sassmann}
\date{\today}
\maketitle

\begin{abstract}
\end{abstract}

\section{Technical Report}

\subsection{Deployment and Development Environment}

In the orginal project proposal we planed to use OpenIndiana,
an illumos based distribution, as your base deployment and development
platform.\footnote{\url{http://wiki.illumos.org/display/illumos/illumos+Home}, \url{http://openindiana.org/}}
While we still consider it a better platform with
the availability of better debugging and analysis tools
such as dtrace and mdb, and a better filesystem in the form of zfs,
we did run into problems connecting various shared-ip zones over a single network interface.
As we don't wanted waste too much critical time at the
start period of the project, we switched to a linux based
distribution and used containers to separate our
services.  For the container management we initially
used pre-stable release of docker.\footnote{\url{https://www.docker.com/}}
That was a slightly
bumpy road, as we encountered various bugs and interface changes.
With the advent of docker version 1.0 the interface has
stabilized and the platform has matured.
We provide docker container build recipes for our
postgres, exist, fuseki and django services.
For the developer we provide a vagrant file which 
installs your deployment distribution (CoreOS)
into a virtual machine and then provisions our
docker containers.\footnote{CoreOS is a custom minimal distribution just to run
containers \url{https://coreos.com/} \href{https://coreos.com/}{CoreOS} its similar to the illumos based SmartOS.}
Thus the developer only needs to run the command: \lstinline´# vagrant up´
to install a local development environment.
Deployment to a server is similar simple by running the docker container build recipes
by hand.

\subsection{Web Application}

The web application provides the primary interface to the user so he can  interact with our services.
The web application connects and interacts with the various data sources and presents
the results to the user.
We currently use the django framework with various additional modules,
the main advantage of a python based framework resides in the availability
of some third-party modules, that easy the interaction with existdb and XML
documents.\footnote{eulxml eulexistdb}
The user is presented with an interface to browse our collection of texts by
author or title.  Other services are also available to the user, like our
OCR or the annotation service.

We consider the git file repository of your TEI documents are the master copy
of our documents.  The main reason be for this stipulation resides in the
UIMA component which annotates new entities on demand, but for which we
currently only have a file collection reader.  If an update to the master copy
is appliedi, our XML database will reload the documents that did change.

\subsubsection{Text Viewer}

The user can browse our TEI documents currently loaded in the XML database by title
or author.  If he desires to view a single document, the TEI document under question
will be fetched from the XML database.  Before the document is present to the
user it must be XTSL transformed to rewrite the TEI tags into valid HTML
tags.

Texts and single chapters can be downloaded in a plain text format or as colored pdf files.
The user has the option to undisplay various annotations he has no interest in, for example
he can hide the (modern) interpuction by pressing ".".

New annotations can also be added in the viewer. The user selects the phrase he wants
to mark with the mouse and then enters a keyboard shortcut to tag the selection.
The marked phrase is send to the server and after a validation enters
our RDF store as a new fact.  Our RDF store is partitioned into 2 named graphs.
New facts supplied by users are by default stored in an extra graph, to seperate them
from the core verified RDF store.  An table with the available keyboard commands
can be found at the documentation section of the website.
Annotations can be viewed and managed in the django admin interface if they are not
automatical added to the RDF store.

\subsubsection{OCR}

Part of our web application is a OCR service.  We use tesseract in the backend for the
actually recognition.  As tesseract is an open source solution the accuracy does not
reach the levels of commercial offerings.  Our experiments with ABBYY FineReader
resulted in a much better overall recognition rate, but the licensing fee is bound
to the pages scaned.\footnote{\url{http://www.abbyy.de/}}
We hope the Internet Archive upgrades they version of ABBYY to provide better fulltexts.

\subsection{SPARQL}

\subsection{UIMA}

\subsection{Further Work}

Single services should be clustered to provide better availability. While CoreOS's etcd key-value store
and fleet provide such capabilities, Mesos or Kubernetes
\footnote{\url{http://mesos.apache.org/} \url{https://github.com/GoogleCloudPlatform/kubernetes}}
% clustering


\section{Research Results}

CDDB
knowledge graph
dbpedia

\end{document}
