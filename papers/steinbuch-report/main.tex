
\documentclass[12pt, draft]{article}

\usepackage{xeCJK}
\setCJKmainfont{AR PL UKai TW}

\usepackage{csquotes}
\usepackage[english]{babel}

\usepackage[notes, backend=biber]{biblatex-chicago}
\bibliography{main}

%\usepackage{parskip}

\usepackage{color}
\usepackage[usenames,dvipsnames]{xcolor}

\usepackage[normalem]{ulem}
\usepackage{listings}
\usepackage{hyperref}

\begin{document}

\title{Karl-Steinbuch-Stipendium}
\author{David Höppner and Manuel Sassmann}
\date{\today}
\maketitle

\begin{abstract}
\end{abstract}

\section{Technical Report}

\subsection{Deployment and Development Environment}

In the orginal project proposal we planed to use OpenIndiana,
an illumos based distribution, as your base deployment and development
platform.\footnote{\url{http://wiki.illumos.org/display/illumos/illumos+Home}, \url{http://openindiana.org/}}
While we still consider it a better platform with
the availability of better debugging and analysis tools
such as dtrace and mdb, and a better filesystem in the form of zfs,
we did run into problems connecting various shared-ip zones over a single network interface.
As we don't wanted waste too much critical time at the
start period of the project, we switched to a linux based
distribution and used containers (LXC) to separate our
services.  For the container management we initially
used pre-stable release of docker.\footnote{\url{https://www.docker.com/}}
That was a slightly
bumpy road, as we encountered various bugs and interface changes.
With the advent of docker version 1.0 the interface has
stabilized and the platform has matured.
But its design is still inferior to other
container implementations like the solaris zones or freebsd jails.

We provide docker container build recipes for our
postgres, exist, fuseki and django services.
For the developer we provide a vagrant file which 
installs your deployment distribution (CoreOS)
into a virtual machine and then provisions our
docker containers.\footnote{CoreOS is a custom minimal distribution just to run
containers \url{https://coreos.com/} \href{https://coreos.com/}{CoreOS} its similar to the illumos based SmartOS.}
Thus the developer only needs to run the command: \lstinline´# vagrant up´
to install a local development environment.
Deployment to a server is similar simple by running the docker container build recipes
by hand.

\subsection{Web Application}

The web application provides the primary interface to the user so he can  interact with our services.
The web application connects and interacts with the various data sources and presents
the results to the user.
We currently use the django framework with various additional modules,
the main advantage of a python based framework resides in the availability
of some third-party modules, that easy the interaction with existdb and XML
documents.\footnote{eulxml eulexistdb}
The user is presented with an interface to browse our collection of texts by
author or title.  Other services are also available to the user, like our
OCR or the annotation service.

We consider the git file repository of your TEI documents are the master copy
of our documents.  The main reason be for this stipulation resides in the
UIMA component which annotates new entities on demand, but for which we
currently only have a file collection reader.  If an update to the master copy
is appliedi, our XML database will reload the documents that did change.

\subsubsection{Text Viewer}

The user can browse our TEI documents currently loaded in the XML database by title
or author.  If he desires to view a single document, the TEI document under question
will be fetched from the XML database.  Before the document is present to the
user it must be XTSL transformed to rewrite the TEI tags into valid HTML
tags.

Texts and single chapters can be downloaded in a plain text format or as colored pdf files.
The user has the option to undisplay various annotations he has no interest in, for example
he can hide the (modern) interpuction by pressing ".".

New annotations can also be added in the viewer. The user selects the phrase he wants
to mark with the mouse and then enters a keyboard shortcut to tag the selection.
The marked phrase is send to the server and after a validation enters
our RDF store as a new fact.  Our RDF store is partitioned into 2 named graphs.
New facts supplied by users are by default stored in an extra graph, to seperate them
from the core verified RDF store.  An table with the available keyboard commands
can be found at the documentation section of the website.
Annotations can be viewed and managed in the django admin interface if they are not
automatical added to the RDF store.

\subsubsection{OCR}

Part of our web application is a OCR service.  We use tesseract in the backend for the
actually recognition.  As tesseract is an open source solution the accuracy does not
reach the levels of commercial offerings.  Our experiments with ABBYY FineReader
resulted in a much better overall recognition rate, but the licensing fee is bound
to the pages scaned.\footnote{\url{http://www.abbyy.de/}}
We hope the Internet Archive upgrades they version of ABBYY to provide better fulltexts.

\subsection{SPARQL}

We also run a public SPARQL (SPARQL Protocol and RDF Query Language) endpoint.
This endpoint can be used via a web interface or programmatically through
a client library.  It provides access to our manually modeled facts
and facts extracted from automatic annotated sources.
We use the functional notation of OWL (Web Ontology Language) to generate
the corresponding RDF (Resource Description Framework) triples.
Currently we employ a java based RDF store.\footnote{\url{http://jena.apache.org/documentation/serving_data/index.html}}
If this solution does not scale well enough we might evaluate
4store as an alternative.\footnote{\url{http://4store.org/}}

\subsection{UIMA}

UIMA (Unstructured Information Management Architecture) started as an
 IBM project (Watson).\footnote{\url{http://uima.apache.org/}}
It was later open sourced and is now an apache project.
UIMA is a architecture and a standard for annotating and enriching unstructured data.
It can be used for textual sources, but its not limited to this domain.
In UIMA various components interact on top of a common analysis structure (CAS)
to annotate documents. At the core we find analysis engines (AE) that
annotate single documents by inserting types with offsets into the CAS.
Analysis engines can share common knowledge (data structures) by
using shared resource objects. This not only bundles
source code into a single place but also speeds up processing as
common data structures will be constructed only once and will be cached
between single documents and by all engines.
Before the analysis engines analyse documents, those documents
must be transformed into the CAS format.
This is done by the collection reader interface. Collection readers
iterate through a document collection and build CAS structures
for every single element. The CAS is passed then on to the analysis engines.
As a final step the CAS could then be serialized again
by a CAS consumer to generate a document in the original document format.
In our UIMA component we follow this workflow but an additional CAS consumer
extractes new information from the annotation it encounters
in the newly processed document.  As analysis engine can be run in
parallel or sequential order, as independent or
depend engines (engine B needs the annotation results from engine A)
schedules are needed. We currently use a simple linear pipeline to
run our analysis engines. UIMA currently provides only this functionality
more complex setups needs additional scale out frameworks that are
not as easy to deploy.




Besides the java version the framework also provides a C++ interface.

\subsection{Further Work}

Single services should be clustered to provide better availability. While CoreOS's etcd key-value store
and fleet provide such capabilities, Mesos or Kubernetes
\footnote{\url{http://mesos.apache.org/} \url{https://github.com/GoogleCloudPlatform/kubernetes}}
% clustering


\section{Date and Time Expressions}

\subsection{Dynasties}
In our context a dynasty is a period of time in which a succession of hereditary
rulers by patrilineal descent from the dynastic founder ruled.\footcite[3]{Wilkinson:2012} 
The probably most famous Chinese dynasty known to the general readership
is the Tang 唐 dynasty which was ruled by the Li 李 clan and lasted from 618 AD to 907 AD.
An overview of all terms for imperial dynasties can be found in the literature.\footnote{
Table 3 in \cite[4]{Wilkinson:2012}}
Most expressions for dynasties in chinese are polysemes. For example the aforementioned
\emph{tang} has also the meanings of 'dike' or it can function as a place name.\footcite[117]{Wang:2000}

\subsubsection{Dynasty + Person}
唐韓愈 Han Yu () of the Tang dynasty

\subsubsection{Dynasty + Expression}
宋興且百年 The Song dynasty flourished since hundred years

宋之中葉


dates, lucky days, travel ->

eclipse of the sun

24 Solar Terms

\section{General Problems}
CDDB
knowledge graph
dbpedia


chenghu expressions
personal names per time periods

\end{document}
